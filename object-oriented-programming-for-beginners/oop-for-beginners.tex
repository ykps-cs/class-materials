\documentclass[12pt]{article}

\usepackage[margin=0.75in, top=1in, bottom=1in, a4paper]{geometry}
\usepackage{amsmath}
\usepackage{amssymb}
\usepackage{amsthm}
\usepackage{pgfplots}
\usepackage{mathtools}
\usepackage{booktabs}
\usepackage{indentfirst}
\usepackage{tikz-cd}
\usepackage{mathtools}
\usepackage[colorlinks, linkcolor=blue]{hyperref}

\theoremstyle{definition}
\newtheorem{defn}{Definition}[section]
\newtheorem{ex}{Exercise}

\definecolor{codegray}{gray}{0.9}
\newcommand{\code}[1]{\colorbox{codegray}{\texttt{#1}}}
\newcommand{\ttype}[1]{\text{#1}}
\newcommand{\gtype}[1]{\textbf{T}_\text{#1}}

\title{\vspace{-2.0cm}Beginners' Guide to Object-Oriented Programming}
\author{Linxuan Ma}

\begin{document}
	\maketitle
	
	\abstract{This guide aims to layout the foundations and prerequisites for a student to proficiently program with object-oriented concepts. This guide intends to have a gradual learning curve and focuses on establishing a solid theoretical foundation before application to alleviate the "WHY THE HELL IS MY COMPILER SCREAMING AT ME" symptom that every programmer faces when starting off. Working knowledge with Java (minus the OOP part) is required.}
	
	\section{Prerequisite: Objects}
	
	The core principle of object-oriented programming, or arguable any form of programming, is to abstract away the tedious implementation details and instead deal with simpler, higher-level concepts. One such example is the concept of function. Similar to its counterpart in mathematics, a function in the scope of computer science is a piece of \emph{sealed} code whose implementation is irrelevant given that we are the caller and we know what it does conceptually. The function's \textit{behavior} is the higher-level concept which we are interested in; the function's implementation detail is the lower-level that we abstract away. Function is an abstraction for the \emph{procedure} of a program.
	
	Similarly, \emph{object} is an abstraction for the data of a program (loosely speaking). Instead of dealing with a plethora of arbitrary attributes and data in a program, we can instead group them into units called \emph{objects}. This pattern allows us to have an array of \code{Person} instead of an abominable pile of \code{int[] age} and \code{String[] name}.
	
	Having a bunch of "grouped" data is pretty useless by itself, as there wouldn't be a simple way to navigate and manipulate them. Therefore, instead of having arbitrary blobs of data with anonymous identity, Java associates each object with a \emph{type}, which dictates the behavior and organization of the data within the object.
	
	\begin{defn}
		An \emph{object} is a container for data, and has the following information:
		\begin{itemize}
			\item Class: a "blueprint" which specifies the behavior of the object
			\item Attributes: information contained in the object as fields
		\end{itemize}
	\end{defn}
	
	The concept of attribute is trivial; they are just what you put as fields in the class declaration of the object, e.g. \code{private int age}. The concept of type, on the other hand, requires much more attention, and is one of the core concept of object-oriented programming.
	
	\section{Theoretical Foundation of Types}
	
	"Type" might be a major source of confusion, only worsened by the more abstract ideas that "type" induces (e.g. polymorphism, variance). To simplify things, we will use simple mathematics concepts to illustrate the relation between types and objects, as well as the relation between types and types.
	
	Types display remarkable resemblance of mathematical sets, and we can treat them as such in the scope of OOP. A type is just a set whose elements are objects. Some examples are:
	\begin{align*}
		\ttype{Integer} &= \{\dots, -1, 0, 1, \dots\} \\
		\ttype{Boolean} &= \{\textbf{true}, \textbf{false}\}
	\end{align*}
	
	We can check if an object is of a type by using the \code{instanceof} operator in Java. Such a check is equivalent to checking if an object is an element in the type's set in our mathematical notion:
	\begin{gather*}
		x \text{ is of type \code{Person}}  \iff x \in \ttype{Person}
	\end{gather*}
	
	\subsection{Methods}
	
	\subsection{Types and Classes}
	
	As mentioned above, each object is associated with a class whose content dictates the object's behavior. A class is a type, among with interface types and primitive types (see \hyperref[sec:primitives]{Remarks 2}). Each object must belong to a class with which the object is instantiated. The class of an object is the immediate type that contains it, i.e. the smallest type set that contains the object.
	
	Apart from class types, there are also interface types. Interface types are the more abstract types that exists to solely denote certain capabilities in the form of \emph{abstract methods}.
	
	\begin{defn}
		An \emph{abstract method} is a method that only possesses a signature (see \hyperref[sec:signature]{Remark 3}) and no actual code. It exists as a requirement such that all classes that implement the method's interface must provide an implementation of the abstract method according to its anticipated behavior.
	\end{defn}
	
	Abstract methods enforces all instance objects to provide an implementation of it. For instance, an abstract method $\text{f}_\text{die}	: \ttype{IAnimal} \to \ttype{Void}$ declared in the interface \code{IAnimal} states that "all \code{IAnimal}s can die". Due to the vast potential difference present in the inheriting classes of \code{IAnimal}, the interface does not know an implementation that satisfies all of them, and therefore leaves the implementation to each class to decide. Nonetheless, each inheriting class must provide a way for the animal to die.
	
	For instance, consider the inheritance hierarchy:
	\begin{equation*}\begin{tikzcd}
		& \ttype{IAnimal}\\
		\ttype{IMarine} \ar[ru] && \ttype{IVertebrate} \ar[lu] \\
		& \ttype{Fish} \ar[lu] \ar[ru]
	\end{tikzcd}\end{equation*}
	
	An object $a \in \ttype{Fish}$ has the class \code{Fish}. Since \code{Fish} inherits \code{IMarine} and \code{IVertebrate}, $a$ is also a member of \code{IMarine} and \code{IVertebrate}, as well as \code{IAnimal}. $a$ is an instance of all four types mentioned above, yet its unique class is \code{Fish}. This shows that an object can be an instance of many types, but only belongs to one class.
	
	Note that even if an object $a \in \gtype{A}$ and $\gtype{A} \subset \gtype{B}$ where $\gtype{A}$ and $\gtype{B}$ are both classes, the class of $a$ is still $\gtype{A}$ (as it is the smallest set containing $a$), and can be witnessed by calling \code{a.getClass()}.
	
	\subsection{Notion of Equivalence}
	
	To construct a set, we must define the notion of equivalence with regards to its elements. How shall we define equivalence over the objects of a type in Java? Unfortunately, there are two kinds of equivalence in Java (consider objects \code{a} and \code{b}):
	\begin{enumerate}
		\item \code{a == b}: compares whether the two references to objects are the same, i.e. if \code{a} and \code{b} refers to the \emph{same} object
		\item \code{a.equals(b)}: a programmer-defined equivalence behavior for the object by overriding \code{Object\#equals}
	\end{enumerate}
	
	To illustrate, \code{new String("hello") == new String("hello")} would be false, as the two references being compared are different objects despite containing the same content. However, \code{new String("hello").equals(new String("hello"))} would be true, as the two strings contain the same content, and thereby satisfy the requirement of equivalence as defined in \code{String\#equals} by the author of the \code{String} class.
	
	For the rest of this guide, equivalence of objects refers to \code{Object\#equals} unless stated otherwise.
	
	\section{Remarks}
	
	\subsection{Function Notation}
	
	\subsection{Primitive Types}
	\label{sec:primitives}
	
	\subsection{Method Signatures}
	\label{sec:signature}
	
\end{document}
